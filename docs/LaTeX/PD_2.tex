\subsection{Iteration to refine $\lambda$}
\label{sec:lambda_iter}

Recall that we want to find a regularization parameter $\l>0$ such that the solution $p(\l)$ of the equation
$(H+\l I)p=-g$ satisfies $\tilde r:=\norm{p(\l)}\approx r$, where $r$ is the value of the trust radius.

Here we have a reasonable guess for a starting value $\l_0=\norm g/\sqrt r$ which however overestimates $\l$
in every case ($\l_0$ is too large). This means that $p(\l)$ will certainly be in $B_r(x)$ but may be very close to the current
iterate $x$, which is undesirable unless this iterate is already close to the minimizer of $f$.
See equation (\ref{lambda_eq_1}) in section \ref{sec:dog_leg_regularized} and subsequent discussion.

With one single eigenvalue or Schur decomposition of $H+\l_0 I$ a cheap iterative algorithm to refine $\l$
can be implemented. However we want to work with Cholesky factorizations (since these are both more accurate 
and much faster than both the Schur decomposition and the Eigenvalue decomposition). Moreover we want to keep the 
number of Cholesky factorizations to a minimum.

In reference \cite{NG} such an iteration for $\l$ is developped which however requires a Cholesky factorization of
$H+\l I$ for each new value of $\l$. We can get the iteration from the Cholesky factorization of 
$$
H+\l I=LL'
$$
as follows:
%
\begin{align}
\label{lambda_iter} 
\l_{next} &= \l+\frac{\norm{p(\l)}-r}{r}\times\frac{\norm{p(\l)}^2}{\norm{w}^2},\q\text{where}\\\notag
Lw &= p(\l).
\end{align}
%
However we do then need a second Cholesky factorization of $H+\l_{next}I$ to compute the new step $p(\l_{next})$.





\section{Line search methods}
\label{sec:line_search}

A robust line search is the backbone of our optimization routines. At the current iterate $x$ we have a direction $p^H$ 
(the full Newton step) and the global minimizer $x^H:=x+p^H$ of the quadratic Taylor approximation  $\ol f$ of the objective 
function $f$. 

As a first step we minimize $f$ on the line segment $[x,x+p^H]$, that is we minimize the function
$$
\phi(t)=f(x+tp^H),\q t\in[0,1].
$$
In principle we have both the first and second derivative of $\phi$ as
$$
\phi'(t)=\nabla f(x+tp^H)\cdot p^H
\q\text{and}\q
\phi''(t)=(p^H)'H(x+tp^H)p^H,
$$
where $H(z)$ is the Hessian of $f$ at $z$. Here the computation of values of $f$ is relatively cheap, 
but the gradient is more expensive and the Hessian is most expensive. For this reason we only discuss methods that make no use
of the second derivative. This rules out the Newton method. 

In practice we will not be able to avoid the use of the first derivative since line search termination conditions are based on 
derivative information.  

\subsection{Methods which do not use derivative information}

We start with methods that use only the values of the objective function $\phi$ but not the derivative $\phi'$.

\subsubsection{Golden Search} 
\label{sec:golden_search}

Suppose we want to minimize a function $\phi=\phi(t)$ on the interval $[0,1]$. The algorithm produces a
sequence triples $0\leq a_k<c_k<b_k\leq 1$. The point $c_k$ splits the current interval $[a_k,b_k]$ which
brackets a minimizer and the search continues in one of the subintervals $[a_k,c_k],[c_k,b_k]\subseteq[a_k,b_k]$.

\noindent
The split maintains the invariant
%
\begin{equation}
\label{inv1}
|I|=\rho|J|,
\end{equation}
%  
where $I$ is the longer and $J$ the shorter of the subintervals $[a_k,c_k],[c_k,b_k]\subseteq[a_k,b_k]$.
The following invariant is also maintained, once it is obtained:

%
\begin{equation}
\label{inv2}
\phi(c_k)\leq\min(\phi(a_k),\phi(b_k))
\end{equation}
%  
This condition ensures that the open interval $(a_k,b_k)$ contains a local minimum of $\phi$. If this
condition is never obtained (boundary minimum), then each triple contains the best minimizer found so far.
The points are spaced in such a fashion that the length of the interval $[a_k,b_k]$ decreases geometrically. 

In the description below we assume that condition (\ref{inv2}) has been obtained, the other case being trivial.
The algorithm proceeds as follows: we split the larger of the two subinterval $[a_k,c_k]$ and $[c_k,b_k]$, 
\textit{wolog} the interval $[c_k,b_k]$ into subintervals $[c_k,d_k]$ and $[d_k,b_k]$. There are now two cases:

\medskip\noindent
(A) $\phi(d_k)\leq\phi(c_k)$. Then we have $\phi(d_k)\leq\min\{\phi(c_k),\phi(b_k)\}$ and we can replace the triple
$a_k<c_k<b_k$ with the triple $c_k<d_k<b_k$.

\medskip\noindent
(B) $\phi(c_k)\leq\phi(d_k)$. Then we have $\phi(c_k)\leq\min\{\phi(a_k),\phi(d_k)\}$ and we can replace the triple
$a_k<c_k<b_k$ with the triple $a_k<c_k<d_k$.

In every case the invariant (\ref{inv2}) is maintained. The question now becomes how we place the point $d_k$.
We have $a_k<c_k<d_k<b_k$. Let
$$
\alpha=c_k-a_k,\ \beta=d_k-c_k\text{ and }\gamma=b_k-d_k
$$
the length of the three subintervals. Since the next interval will either be $[a_k,d_k]$ or $[c_k,b_k]$ it is 
reasonable to require that these have the same length, i.e. $\alpha+\beta=\beta+\gamma$ from which we obtain
$$
\alpha=\gamma.
$$
Next we require that the proportion in the spacing between the points be maintained by passing from 
the triple $(a_k,c_k,b_k)$ to the triple $(c_k,d_k,b_k)$, i.e.: 
$$
\frac{\alpha}{\beta+\gamma}=\frac{\beta}{\gamma}
\q\text{i.e.}\q
\frac{\alpha}{\beta+\alpha}=\frac{\beta}{\alpha}:=\rho.
$$
Note that this implies $\alpha>\beta$ and thus $\rho=\beta/\alpha$ is also the corresponding ratio for the triple 
$(a_k,c_k,d_k)$. It follows that this ratio is maintained as an invariant of the algorithm in both cases (A) and (B). 

\noindent
Dividing by $\alpha$ we obtain $\rho=1/(1+\rho)$, i.e. $\rho^2+\rho-1=0$ from which 
$$
\rho=(\sqrt{5}-1)/2
$$
is the golden ratio. Let us note in passing that the relation $\rho=1/(1+\rho)$ implies that all the continued 
fractions $[1,1,\dots,1,\rho]$ are equal to $\rho$. To see this, simply replace $\rho$ in the fraction $1/(1+\rho)$
with $1/(1+\rho)$ and continue such replacements.

\noindent
Note that the spacing of the point $c_k$ is given by the fraction
$$
\frac{b_k-c_k}{b_k-a_k}=
\frac{\beta+\gamma}{\alpha+\beta+\gamma}=
\frac{\alpha+\beta}{\alpha+\beta+\alpha}=
\frac{1}{1+\alpha/(\alpha+\beta)}=
\frac{1}{1+\rho}=\rho.
$$

The new interval is $[c_k,b_k]$ or $[a_k,d_k]$, both of equal length.
Thus the interval length $b_k-a_k$ decreases geometrically by a factor of $\rho=0.618034$ in each step. 
We have seen above that $\rho=\alpha/(\beta+\gamma)$ and so
$$
\frac{c_k-a_k}{b_k-c_k}=\frac{\alpha}{\beta+\gamma}=\rho=\frac{b_k-c_k}{b_k-a_k}
$$
i.e.: the fraction of the shorter to the longer subinterval = the fraction of the longer subinterval to the whole.
This is the rule of the Golden Ratio and that is where this method derives its name.

In this description we have \textit{assumed} that $[c_k,b_k]$ is the longer subinterval of $[a_k,b_k]$.
But not in general will the longer subinterval be the interval on the right. Suppose for example that we pass to the
triple $(a_k,c_k,d_k)$. We have seen above that $\alpha>\beta$, i.e. in this triple $[a_k,c_k]$ is the longer subinterval
and is the interval that has to be split in the next step.

\medskip\noindent
\textbf{Algorithm:} start with $a_k=0$, $b_k=1$ and set $c_k=\rho$. Given any triple $(a_k,c_k,b_k)$
where the point $c_k$ splits the interval $[a_k,b_k]$ such that 
$$
\text{longer subinterval}=\rho\times\text{full interval}
$$
split the longer subinterval with the point $d_k$ in the same proportion and such that these proportions
are maintained in both subsequent triples (this determines if the shorter subinterval is on the right or
on the left).

This gives us two new triples of points $(a_k,c_k\wedge d_k, c_k\vee d_k)$ and $(c_k\wedge d_k, c_k\vee d_k, b_k)$.
Move to the triple which contains the best minimizer so far. 

\subsubsection{Quadratic Interpolation}
\label{sec:quadratic_approx}
 
Given a triple of points $a<c<b$ with associated values $\phi(a)$, $\phi(b)$ and $\phi(c)$ we now approximate 
$\phi$ with a quadratic polynomial $P(t)$ which interpolates the points $(a,\phi(a))$, $(b,\phi(b))$ and $(c,\phi(c))$,
then minimize this polynomial for a final candiate point for the minimum. Finally we retain the point at which the
smallest value of $\phi$ so far occurs.

This is suitable for mixing with the strategy of Golden Search since at each iteration of Golden search we have exactly 
such a triple of points.

\noindent
Note that $P(t)$ has the form
$$
P(t)=
\phi(a)\frac{(t-b)(t-c)}{(a-b)(a-c)}+
\phi(b)\frac{(t-a)(t-c)}{(b-a)(b-c)}+
\phi(c)\frac{(t-a)(t-b)}{(c-a)(c-b)}.
$$
from which it follows that
$$
P'(t)=
\phi(a)\frac{2t-(b+c)}{(a-b)(a-c)}+
\phi(b)\frac{2t-(a+c)}{(b-a)(b-c)}+
\phi(c)\frac{2t-(a+b)}{(c-a)(c-b)}.
$$
Setting the derivative equal to zero yields
%
\begin{align*}
t &= N/D,\q\text{where}\\
N &= 
\phi(a)\frac{b+c}{(a-b)(a-c)}+
\phi(b)\frac{a+c}{(b-a)(b-c)}+
\phi(c)\frac{a+b}{(c-a)(c-b)}\q\text{and}\\
D &= \phi(a)\frac{2}{(a-b)(a-c)}+
\phi(b)\frac{2}{(b-a)(b-c)}+
\phi(c)\frac{2}{(c-a)(c-b)}.
\end{align*}
%
Here we need to check that $P''(t)>0$ to verify that the interpolating polynomial $P$ does assume a minimum. 
To this end we note that $P''(t)=D$, for all $t$. Note that the above formulas can be simplified by multiplying
both $N$ and $D$ with $(a-b)(a-c)(b-c)>0$ to obtain
%
\begin{equation}
\label{q2_min} 
t=\frac{1}{2}\times
\frac{
\phi(a)(b^2-c^2)+\phi(b)(c^2-a^2)+\phi(c)(a^2-b^2)}
{\phi(a)(b-c)+\phi(b)(c-a)+\phi(c)(a-b)}.
\end{equation}
%
and we note that $P''(t)$ has the same sign as the new denominator.


\subsection{Methods which use the first derivative}

Now we move on to  methods which use the first derivative
$$
\phi'(t)=\nabla f(x+tp^H)\cdot p^H.
$$

\subsubsection{Quadratic Interpolation}
\label{sec:quadratic_approx_deriv}

With derivative information we need only two points to determine the quadratic interpolating polynomial.
Say we have points $a,b$ and associated values $\phi(a)$, $\phi(b)$ as well as the derivative $\phi'(a)$.
Then there is a unique quadratic polynomial $P(t)$ which interpolates $\phi$ at the points $(a,\phi(a))$ and $(b,\phi(b))$
and which satisfies $P'(a)=\phi'(a)$. We then minimize this polynomial to obtain a new candidate point.

\noindent
Develop $P$ centered at $a$ to obtain
$$
P(t)=\phi(a)+\phi'(a)(t-a)+q(t-a)^2,
$$
where the coefficient $q$ is determined from the equation $P(b)=\phi(b)$ yielding
$$
q=\frac{\phi(b)-\phi(a)-\phi'(a)(b-a)}{(b-a)^2}
$$
and setting the derivative $P'(t)=0$ we obtain the location of the minimum as 
%
\begin{equation}
\label{q2_deriv_min}
t = a-\frac{\phi'(a)}{2q}.
\end{equation}
% 
Here we need to check that $q>0$ to ensure that the interpolating polynomial does in fact have a minimum.


\subsection{Trust radius from line search}
\label{sec:line_search_trust_radius}

Since the line search is relatively cheap we can try to derive a guess for the trust radius based on a study of the 
objective function $f$ on the line segment $[x,x+p^H]$. Clearly the global minimizer $x+p^H$ of the quadratic approxiation
$\ol f$ of $f$ also minimizes $\ol f$ on this line segment. Actually the quadratic approximation $\ol f(x+tp^H)$ is \textit{decreasing}
on this entire line segment.

To see this recall that $p:=p^H$ is the solution of $Hp=-g$, where $g=\nabla f(x)$ and $H=\nabla^2f(x)$ is the hessian of $f$ at the current
iterate $x$. With this the quadratic approximation $\ol f$ has the form
$$
\ol f(x+h)=f(x)+g'h+\frac{1}{2}h'Hh
$$
and especially for $h=tp$, $t\in[0,1]$ we obtain
$$
\ol f(x+tp)=f(x)+tg'p+\frac{1}{2}t^2p'Hp
$$
from which, using $Hp=-g$,
$$
\frac{d}{dt}\ol f(x+tp)=g'p-tg'p=(1-t)g'p
$$
However it is easy to see that $g\cdot p<0$ by writing both $g$ and $p$ as linear combinations of the eigenvectors of $H$.
Indeed this is true for all the steps $p(\l)$ computed as the solutions of the regularized Newton equation $(H+\l I)p=-g$,
where $\l\geq 0$. Indeed, if $Hu_j=\l_ju_j$ and 
$$
g=\sum_ja_ju_j
$$ 
then the solution $p(\l)$ of $(H+\l I)p=-g$ is given by
$$
p(\l)=-\sum\frac{a_j}{\l_j+\l}u_j,
$$
where all the eigenvalues $\l_j$ are strictly positive (convexity of $f$).  It readily follows that
$$
g'p(\l)=g\cdot p(\l)=-\sum_j\frac{a_j^2}{\l_j+\l}<0,
$$
i.e every step $p(\l)$ is a step in a descent direction of $f$ and we have
%
\begin{align*}
\frac{d}{dt}\ol f(x+tp(\l))&=
g'p(\l)+tp(\l)'Hp(\l)\\&=
g'p(\l)+tp(\l)'(H+\l I)p(\l)-t\l\norm{p(\l)}^2\\&=
(1-t)g'p(\l)-t\l\norm{p(\l)}^2=
g'p(\l)-t(g'p(\l)+\l\norm{p(\l)}^2).
\end{align*}
%
Assume now that $\l\geq 0$ is so small that $g'p(\l)+\l\norm{p(\l)}^2$ is still negative. Then 
%
\begin{equation}
\label{line_decrease}
\frac{d}{dt}\ol f(x+tp(\l))\leq 0,\q\text{for all }0\leq t\leq\frac{g'p}{g'p+\l\norm{p(\l)}^2}.
\end{equation}
%
So for $\l=0$, where $p(\l)=p^H$ is the full Newton step, this derivative is negative, and hence the 
quadratic approximation $\ol f$ decreasing, on the full interval $[0,1)$. For the steps $p(\l)$
this holds on an even larger interval $[0,1+\epsilon)$ but these steps are smaller than the full Newton step.
This means that the quadratic approximation will always assume its minimum at the point $x+p$ on the line segment
$[x,x+p]$.

Now this is not necessarily true also for the objective function $f$ itself and this is where we can get information 
about the trust radius: the maximum value for the trust radius would be the size $r=\norm{p}$ of the full step $p$. 
However if we detect that the objective function $f(x+tp)$ assumes its minimum at a point $t<1$, then the trust radius 
needs to be reduced below the value $r=t\norm{p}$. 

\noindent
This applies to the full Newton step $p=p^H$ as well as to the regularized Newton steps $p=p(\l)$ and suggests 
the following algorithm: start off with a small value of $\l$, e.g. $\l=\min(0.001,\norm{g}/10)$ and compute the 
corresponding point $p(\l)$. Now conduct golden search on the line segment $[x,x+p(\l)]$. 

At every triple $a<c<b$ the quadratic approximation $\ol f$ satisfies $\ol f(a)>\ol f(c)>\ol f(b)$ and the expectation 
is that the objective function $f$ satisfies the same inequalities. So long as this is the case we continue with 
golden search, always moving to the right (choose the triple containing the current minimizer $b$).

If the inequality $f(a)>f(c)>f(b)$ is maintained for three or four iterations, the trust radius $r$ is set equal
to $\norm{p(\l)}$. 

\noindent
When these inequalities fail for the first time there are two possibilities:

\smallskip\noindent
(A) $f(a)$ is the smallest value. We reduce the trust radius to $r=a\norm{p(\l)}$ and refine $\l$ as described in
section \ref{sec:lambda_iter}.

\smallskip\noindent
(B) $f(c)$ is the smallest value. Then we know that the quadratic polynomial interpolating the points
$(a,f(a))$, $(b,f(b))$ and $(c,f(c))$ must have its minimum in the open interval $(a,b)$. We compute the location
of this minimum as described in section \ref{sec:quadratic_approx} and either continue to narrow down the location 
of the local minimum or reduce the trust radius immediately to the value $r=t_*\norm{p(\l)}$, where $t_*$ is the 
location of the current minimum in the interval $(0,1)$.






 
