\chapter{Primal-Dual Methods}

In this section we consider methods which compute the primal and dual variables
(Lagrange multipliers) simultaneously. We start with the case of a quadratic objective function.

\section{Quadratic programming}

\subsection{Introduction}
\label{subsec:qp_intro}

The minimization of quadratic functions will be an important tool for the minimization of
general nonlinear function $f$. This comes about as follows: the minimization of a general,
$C^2$-function $f$ proceeds by iteration. We start from some point $x_0$ and proceed through 
a series of iterates $x_k$ hopefully convergent to the minimizer $x_*$ of $f$. 

Given iterate $x_k$ we want to find the step $p_k:= x_{k+1}-x_k$ that gets us to the next  
iterate $x_{k+1}$. To do this we approximate $f$ by the quadratic Taylor polynomial centered at
the current iterate $x_k$
%
\begin{equation}
\label{TP2}
\ol f(x_k+p) = f_k + g_k'p+\tfrac{1}{2}p'H_kp,
\end{equation}  
%
where $f_k:=f(x_k)$, $g_k=\nabla  f(x_k)$ is the gradient and $H_k=D^2f(x_k)$ the Hessian of $f$ 
at $x_k$ which is assumed to be \textit{positive definite}. This approximation is accurate only 
locally and hence we trust it only on a ball
$$
B_r(0)=\{\ths p\in\bbR^n\ths:\norm p\leq r\ths\}
$$
of some sufficiently small radius $r$. This radius $r=r_k$ will be chosen using some heuristics 
in each step.

With this we want to minimze the quadratic function $h(p)=\ol f(x_k+p)$ on the ball $B_r(0)$
and let $p_k$ be that minimizer. With this $x_{k+1}$ becomes the minimum of the quadratic 
approximation $\ol f$ on the ball $B_r(x_k)$ and is guaranteed to satisfy 
$\ol f(x_{k+1})\leq \ol f(x_k)=f_k=f(x_k)$.

However $x_{k+1}$ is not necessarily the locus of the minimum of $f$ on the ball $B_r(x_k)$.
To check whether we have made sufficient progress we must evaluate $f(x_{k+1}$ and check if we   
have a sufficient decrease from the value of $f(x_k)$.

From the study of iterative solutions based on line searches (Wolf-condition) we know that it
is sufficient to decrease the value of $f$ by a constant multiple of the optimal decrease 
\textit{of the linear approximation}
%
\begin{equation}
\label{TP1}
\tilde f(x_k+p) = f_k + g_k'p,
\end{equation}  
%
as this will imply convergence to a minimizer possibly requiring a very large number of steps.
This motivates the definition of the 

\subsubsection{Cauchy point}
\label{subsubsec:CauchyPoint} 

Note that $g_k=\nabla f(x_k)$ is also the gradient $\nabla\ol f(x_k)$ of the quadratic 
approximation $\ol f$ at the point $x_k$. The \textit{Cauchy point}  $x^C_{k+1}(r)$ is defined 
as the minimizer $x$ \textit{of the quadratic approximation} $\ol f$ on the line $x=x_k-tg_k$ 
of steepest descent from the point $x_k$ subject to the condition $\norm{x-x_k}\leq r$:
%
\begin{align}
\label{Cauchy point}
x^C_{k+1}(r) &:= x_k-t_*g_k,\q\text{where}\\
t_* &= argmin_{t\geq 0}\ths\ol f(x_k-tg_k),\q\text{subject to }\norm{tg_k}\leq r.
\end{align}
%
Here we assume that $g_k=\nabla f(x_k)\neq 0$ since the minimization algorithm terminates 
as soon as a zero gradient is encountered. Clearly this is equivalent to
$$
t_* = argmin_{t\in[0,r/\norm{g_k}}g(t)\q\text{where }g(t)=\ol f(x_k-tg_k).
$$
Since we are minmizing the quadratic approximaton $\ol f$ and not the function $f$ itself,
$t_*$ and hence the Cauchy point can be found explicitly: we have
$$
g(t)=f_k-t\norm{g_k}^2+\tfrac{1}{2}t^2g_k'H_kg_k
$$
This polynomial has its global minimum at $t_0=\norm{g_k}^2/g_k'H_kg_k$ and $g(t)$ is strictly 
decreasing from $t=0$ to this point. Thus to observe the condition $t\leq r/\norm{g_k}$ we can
simply cut off $t$ at this bound if the global solution $t_0$ is too large:
%
\begin{equation}
\label{t_*}
t_* = \min\left\{\ths
\frac{r}{\norm{g_k}},\ths\frac{\norm{g_k}^2}{g_k'H_kg_k}\ths\right\}.
\end{equation}
%    
With this the condition for the sufficient 
decrease in the value of the objective function $f$ then has the form
%
\begin{equation}
\label{sufficient_decrease}
f(x_k)-f(x_{k+1})\geq\kappa(f(x_k)-f(x^C_{k+1}(r))).
\end{equation}  
%
In the development above the objective function $f$ has been replaced with its quadratic 
approximation $\ol f$ (second order Taylor polynomial) and each step of the iterative
algorithm only depends on this approximation (centered at the current iterate). 

We can therefore abstract from this setting of iterative solution and study the quadratic 
problem in isolation.


\section{Quadratic minimization}
\label{sec:QP}

As a warmup we start with quadratic minimization under equality constraints as this will 
introduce some of the important notions. Thereafter we turn to quadratic approximation on 
a ball (the trust region).


\subsection{Quadratic problem with equality constraints only}
\label{subsec:QP_with_eqs}

Consider first the case where there are no inequality constraints (and hence no
complementary slackness conditions in the first order KKT equations). The problem reads
%
\begin{align}
\label{qp}
? &= argmin_x\ths f(x) = argmin_x\ths c'x+\tfrac{1}{2}x'Gx\q\text{subject to}
\\\label{lc}
Ax &= b,  
\end{align}
%
where $G$ is a symmetric $n\times n$-matrix, $A$ an $m\times n$-matrix with 
$1\leq m<n$ rows, $x\in\bbR^n$ and $b\in\bbR^m$.

We assume that $A$ has full rank and hence $dim(ker(A))=n-m$. Let $Z$ be a 
kernel matrix for $A$, that is the columns of $Z$ form a basis for the kernel 
of $A$. Then $Z$ is an $n\times(n-m)$ matrix of full rank satisfying $AZ=0$.

We say that $G$ is positive semidefinite on the kernel of $A$ if we have
%
\begin{equation}
\label{psd_on_kerA}
v'Gv\geq 0,\q\forall v\in ker(A),
\end{equation}  
%
equivalently if the matrix $Z'GZ$ is positive semidefinite. (Strict) positive 
definiteness of $G$ on the kernel of $A$ is defined analogously and is 
equivalent to the matrix $Z'GZ$ being positive definite.

Note that the kernel of $A$ describes the degrees of freedom in the problem  
since the solution set of the equality constraint $Ax=b$ has the form 
$x_0+ker(A)$, where $x_0$ is any particular solution of this constraint.

\noindent
Denoting the Lagrange multipliers for the equality constraints with $y$, the
Langrangian has the form 
$$
L(x,y)=c'x+\tfrac{1}{2}x'Gx+y'(Ax-b)=
c'x+\tfrac{1}{2}x'Gx+\sum\nolimits_jy_j(a_i'x-b_i),
$$
where $a_i=row_i(A)$, and setting all partial 
derivatives equal to zero yields the KKT conditions for the problem (\ref{qp}), (\ref{lc}) as
%
\begin{equation}
\label{kkt_qp}
\begin{pmatrix}
G & A'\\
A & 0
\end{pmatrix}
\begin{pmatrix}
x\\
y
\end{pmatrix}
=
\begin{pmatrix}
-c\\
b
\end{pmatrix}
\end{equation}  
%
In general the existence of such a Lagrane multiplier $y$ is only a necessary
condition for $x$ to be a local minium under the constraint but here it is
sufficient for a global minimum
%
\begin{prop}
\label{prop:kkt_qp}
Suppose that $G$ is positive semidefinite definite on the kernel of $A$ and 
$x\in\bbR^n$. If there exists a Lagrange multiplier $y\in\bbR^m$ such that
(\ref{kkt_qp}) is satisfied, then $x$ is a solution of  the problem (\ref{qp}),
(\ref{lc}). 

If $G$ is strictly positive semidefinite on $ker(A)$ then $x$ is the unique
solution of (\ref{qp}), (\ref{lc}). 
\end{prop}
%
\textbf{Proof.}
Assume that $(x,y)$ satisfy (\ref{kkt_qp}) and let $w\in\bbR^n$ with $Aw=b$ be
arbitrary. We have to show that $x'Gx+2c'x=2f(x)\leq wf(w)=w'Gw+2c'w$. 

\noindent
Set $v=w-x$. Then $v\in ker(A)$. From (\ref{kkt_qp}) we have 
$v'(Gx+c)=-v'A'y=-y'Av=0$. It follows that
%
\begin{align*}
2f(w) &= 
(x+v)'G(x+v)+2c'(x+v) =  
x'Gx+2c'x+2v'(Gx+c)+v'Gv\\&=
2f(x)+v'Gv\geq 2f(x),
\end{align*}
%
since $G$ is positive semidefinite on $ker(A)$. 
If $G$ is strictly positive semidefinite on $ker(A)$, 
then $w\neq x$ implies $v\neq 0$ and hence $v'Gv>0$,
thus $f(w)>f(x)$ and the claim follows.\qed

\noindent
\textbf{Step to solution.}
Assume that we are starting at some point $x$ and want to compute a 
\textit{step} $\Delta x$ such that $(x+\Delta x,y)$ 
satisfies the above equations. I.e. here $x$ is considered known and
$\Delta x$ and $y$ are the new unknowns.

Substituting $x+\Delta x$ into (\ref{kkt_qp}) and moving the  
known terms to the right we obtain the following system for the step $\Delta x$:
%
\begin{equation}
\label{kkt_qp_step}
\begin{pmatrix}
G & A'\\
A & 0
\end{pmatrix}
\begin{pmatrix}
\Delta x\\
y
\end{pmatrix}
=
\begin{pmatrix}
-(Gx+c)\\
b-Ax
\end{pmatrix}.
\end{equation}  
%
In the present setting there is no need to formulate the problem in this way
since we can just as well solve (\ref{kkt_qp}) directly. However we will use the
quadratic function above as a local approximation to a nonlinear function and 
then an iterative approach is necessary. Naturally then we start at an iterate
$x$ and want to compute a step $\Delta x$ to move to the next iterate.

The matrix on the left of (\ref{kkt_qp_step}) is symmetric but \textit{never}
even positive semidefinite.
However if $G$ is (strictly) positive definite on the kernel of $A$ then this 
matrix is \textit{nonsingular}. Indeed assume that
%
\begin{align*}
Gx+A'y&=0\q\text{and}\\
Ax&=0.
\end{align*}
% 
Then $x\in ker(A)$ and $0=x'(Gx+A'y)=x'Gx+y'Ax=x'Gx$ which implies $x=0$. It 
follows that $y=0$ since $A'$ has full rank. We can thus solve the system
(\ref{kkt_qp_step}) directly with an $LDL'$ decomposition or we can bring it
to block diagonal form as follows: multiply the first equation from the left 
first by $G^{-1}$, then by $A$ to obtain
%
\begin{align*}
A\Delta x+AG^{-1}A'y &= -(Ax+AG^{-1}c)\\
A\Delta x = b-Ax 
\end{align*}  
%
which yields the equation
%
\begin{equation}
\label{kkt_qp_y}
AG^{-1}A'y = -(b+AG^{-1}c).
\end{equation}  
%
Since $A$ has full rank we have $ker(A')=0$ and it follows that the matrix
$AG^{-1}A'$ is also positive definite. Consequently (\ref{kkt_qp_y}) can be   
solved for $y$ using the Cholesky decomposition of $H=AG^{-1}A'$ which has
size $m\times m$ where usually $m<<n$. 

The computation of $G^{-1}A'$ and $G^{-1}c$ can be handled as the solution of
$GX=[A',c]$ using the Cholesky factorization of $G$. Assuming an optimized 
computation of the matrix product $AG^{-1}A'$ by blocking (LaPack, optmized 
BLAS) this will be faster than the direct approach in all cases.

\noindent
\textbf{Nullspace method.} The preceeding approach needed the matrix $G$ to be
invertible (hence positive definite) but Lemma (\ref{prop:kkt_qp}) only needed 
$G$ to be positive definite on the null space of $A$, i.e. the matrix 
$Z'GZ$ is positive definite, where $Z$ is any $n\times(n-m)$ matrix $Z$ such 
that the columns of $Z$ are a basis for the null space of $A$.

Suppose we have such a matrix $Z$. Then the solution set of the constraint
$Ax=b$ has the form $x=x_0+Zz$, $z\in\bbR^{n-m}$, where $x_0$ is any particular
solution of this constraint. Rewriting $f(x)$ in terms of the new variable $z$
we obtain
%
\begin{align*}
f(x)&=\ol f(z)=
\tfrac{1}{2}(x_0+Zz)'G(x_0+Zz)+c'(x_0+Zz)\\&=
\tfrac{1}{2}z'(Z'GZ)z+z'Z'Gx_0+z'Z'c+x_0'Gx_0+c'x_0\\&=
\tfrac{1}{2}z'Hz+d'z+const,
\q\text{where }H=Z'GZ\text{ and }d=Z'(Gx_0+c),
\end{align*}
%
so that the problem (\ref{qp}), (\ref{lc}) is equivalent to the uncontrained problem
%
\begin{equation}
\label{nspace_qp}
? = argmin_z\ths\tfrac{1}{2}z'Hz+d'z\q\text{with }H=Z'GZ\text{ and }d=Z'(Gx_0+c)
\end{equation}
%
the solution of which is the solution of the equation
%
\begin{equation}
\label{nspace_qp_sol}
(Z'GZ)z=Hz=-d
\end{equation}
%
which is an $(n-m)\times(n-m)$ system that can be solved with a Cholesky 
factorization of the matrix $H=Z'GZ$. This matrix product is fast if optimized by
blocking. But the computation of $Z$ is expensive: we can get it from a 
$QR$-decomposition of $A'$:
$$
A' = QR
$$
with $Q$ $n\times m$ orthogonal and $R$ $m\times m$ upper triangular with no 
zeros on the diagonal (since $A'$ has full rank $m$). With this
it is clear that $ker(A)^\perp = Im(A') = Im(Q)$ is the span of the columns of 
$Q$. 

Thus we can get an ON-basis for the kernel of $A$ by enlarging the columns of 
$Q$ to an ON-basis of $\bbR^n$ and $Z$ then consists of the newly added columns.
Some factorizations in LaPack yield such a decomposition
$A'=QR$, where $Q$ is $n\times n$ orthogonal and $R$ $n\times m$ upper 
triangular with the last $n-m$ rows equal to zero ("full" decomposition).

From this we can read off $Z$ directly as $Z=Q[,(m+1):n]$ (the last $n-m$ 
columns of $Q$. However this $QR$ factorization of $A'$ is more expensive than
a Cholesky decomposition.
 

\subsection{Quadratic optimization on a ball}
\label{subsec:QP_on_ball}

Now we consider the problem
%
\begin{align}
\label{qp_on_ball}
? &= argmin_p\ths f(p)\q\text{subject to }
\norm{p}\leq r,\q\text{where }
f(p) = g'p+\tfrac{1}{2}p'Hp
\end{align}  
%
with $H$ $n\times n$ symmetric and positive semidefinite. It is easy to see that
an unconstrained global minimum exists if and only if $g\in ker(H)^\perp$. 
However a global minimum under the constraint always exists.

\noindent
Minimization problems of this sort will be important for trust region methods.

An uncontrained global minimizer $x$ would have to satisfy the equation 
$\nabla f(p)=0$, i.e. $Hp=-g$ which has a solution only if 
$g\in Im(H)=Im(H')=ker(H)^\perp$. Suppose we regularize this equation by replacing
the matrix $H$ with $H+\lambda I$, where $\lambda>0$. This matrix is positive
definite so a solution $x_*$ of 
%
\begin{equation}
\label{H_reg}
(H+\lambda I)p=-g
\end{equation}
% 
always exists and is the unique global minimizer of the function
$$
f_\lambda(p)=p'(H+\lambda I)p+g'p=f(p)+\lambda\norm p^2.
$$
We claim that $p_*$ is an absolute minmizer of the original objective function 
$f$ on the ball $B_r(0)=\{\ths p\in\bbR^n\ths\norm p\leq r\ths\}$ with radius
$r=\norm{p_*}$. Indeed if $v\in B_r(0$ satisfies $f(v)<f(p_*)$ then we would have
$$
f_\lambda(v)=
f(v)+\lambda\norm v\leq 
f(p_*)+\lambda r=
f(p_*)+\lambda\norm{p_*}=f_\lambda(p_*)
$$
contradicting the fact that $p_*$ is a global minimizer of $f_\lambda$. We record this fact as
%
\begin{prop}
\label{prop:f_eps_min}
The unique solution $p_*$ of (\ref{H_reg}) is the global minimizer of 
$f(p)=g'p+\tfrac{1}{2}p'Hp$ on the ball $B_r(0)$ with radius $r=\norm{p_*}$.\qed 
\end{prop}
%
From this it would seem that the solution of (\ref{qp_on_ball}) should be simple
but this is not the case as the radius $r$ in the proposition above 
\textit{depends on the solution} $p_*$.

The main application will be to trust region methods as follows: the 
optimization of a general nonlinear function proceeds iteratively moving in
small steps from one iterate to the next. At each step we use a quadratic 
approximation of the objective function centered at the current iterate $x_k$ to 
compute the next step to be taken from this iterate.

We trust this approximation only on a ball $B_r(x_k)$ of small radius $r$ 
centered at the current iterate $x_k$ and therefore need to have some control 
over that radius. We then minimize the quadratic approximation on that ball 
$B_r(x_k)$ and move in the direction of the minimizer $x_*$ in the next step.
By simple translation $g(p)=f(x_k+p)$ this can be reformulated as a minimization 
over a ball centered at zero as above. With this the variable $p$ is the step to
be taken from the iterate $x_k$.

For this we need some control over the radius $r$ (which will be increased or
decreeased depending on the performance in the previous step: how much the value
of the actual objective function was decreased).

This radius $r=r(\lambda)=\norm{p_*(\lambda)}$ depends on $\lambda$ and we 
need to study this dependence. We can do this using the eigenvector 
decomposition $H=UDU'$ with $U$ orthogonal, $D\geq 0$ a diagonal matrix 
with the eigenvalues $\l_1\leq \l_2\leq\dots\leq\l_n$ of $H$ on the diagonal
and the columns $u_j=col_j(U)$ the corresponding eigenvectors: $Gu_j=\l_ju_j$. 

\noindent
Now expand $g$ as a linear combination of these eigenvectors $u_j$:
$$
g=\sum_ja_ju_j,
$$
where the coefficients $a_j$ are the inner products $a_j=g'u_j$. It then follows
that the $u_j$ are also the eigenvectors of the matrix function $(H+\l I)^{-1}$
with associated eigenvalues $(\l_j+\l)^{-1}$ and this implies that
$$
p_*(\l)=
(H+\l I)^{-1}\sum_ja_ju_j=
\sum_j\frac{a_j}{\l_j+\l}u_j
$$
from which it follows that
%
\begin{equation}
\label{norm_r}
r(\lambda)^2=
\norm{p_*(\lambda)}^2=\sum_j\frac{(g'u_j)^2}{(\l_j+\l)^2}
\end{equation}
%
With this we can then easily solve the equation $r(\lambda)=r$ for $\lambda$ to obtain 
any desired radius $r$ (e.g. using Newton's algorithm) but the computation of the eigenvalue 
decomposition of $H$ is too expensive and has to be avoided. \cite{NW} has an algorithm that   
uses only a small number of Cholesky factorizations (algorithm 4.3, chapter 4, p87) but the  
derivation is unclear.

\noindent
Recall from (\ref{t_*}) in the introduction \ref{subsec:qp_intro} (with $x=0$) that the Cauchy 
point $p^C_r$ of (\ref{qp_on_ball}) is given by
%
\begin{equation}
\label{CauchyPoint_1}
p^C_r = -t_*g
\q\text{where }
t_* = \min\left\{\ths
\frac{r}{\norm{g}},\ths\frac{\norm{g}^2}{g'Hg}
\ths\right\}.
\end{equation}
%  
We now discuss several methods to improve on the Cauchy point:

