\subsubsection{The Dog Leg Method}

The Cauchy point above is the global minimizer $x$ of the quadratic function $f$ along 
the line $x=-tg$, $t\geq 0$, in the steepest descent direction $-g=-\nabla f(0)$ subject 
to the condition $\norm x\leq r$.

In the dog leg method we extend this line by adding a line segment from the Cauchy point 
to the global minmizer $p^H=-H^{-1}g$ of the quadratic function $f$, all subject to remaining
inside the ball $B_r(0)$.

The \textit{dog leg minimizer} $p^D_r$ is the global minimizer of the quadratic function $f$ 
on this extended path to the boundary of the ball $B_r(0)$. If the Cauchy point $p^C_r$ is
already on the boundary of the ball $B_r(0)$ (i.e. if $t_*=r/\norm{g}$), then $p^D_r=p^C_r$
(as no extension is possible).

Otherwise we minimize $f$ also on the line $p=p^C_r+t(p^C_r-p^H)$ subject to the constraint
$\norm p\leq r$ and choose the new minimizer if it is better than the Cauchy point. In fact
it is guaranteed to be better than the Cauchy point, as the function
$$
h(t)=f(p^C_r+t(p^C_r-p^H))
$$ 
is strictly decreasing for $t\in[0,1]$. Moreover the function
$q(t)=\norm{p^C_r+t(p^C_r-p^H)}$ is strictly increasing (see the next lemma).
This implies that the minimum is assumed on the boundary of the ball $B_r(0)$, i.e. $t$ is given
by the equation
%
\begin{align}
\label{t_dog_leg}
r^2 &= \norm{a+t(b-a)}^2=t^2\norm{b-a}^2+2ta'(b-a)+\norm a^2,\q\text{where}\\\notag
a &= p^C_r\q\text{and}\q b = p^H_.
\end{align}
%
Alltogether we have
%
\begin{equation}
\label{DoglegPoint}
p^D_r= 
\begin{cases}
p^C_r&\text{if }\norm{p^C_r}=r,\text{ i.e. if }t_*=-r/\norm g,\\
p^C_r+t(p^H-p^C_r),\q&\text{with }t\geq0\text{ in }(\ref{t_dog_leg})\text{ otherwise}.
\end{cases}
\end{equation}
This is justified by the following
%
\begin{lem}
\label{lem:dogleg}
Assume that $H$ is positive definite. Then\\
(a) The function $q(t)=\norm{p^C_r+t(p^C_r-p^H)}$ is strictly increasing for $t\in[0,1]$.\\
(b) The function $h(t)=f(p^C_r+t(p^C_r-p^H))$ is strictly decreasing for $t\in[0,1]$.
\end{lem}
% 

\medskip\noindent
\textbf{Proof.} Set $a=p^C_r=-t_*g$ and $b=p^H$ and note that $Hb=-g$.
With this we have
%
\begin{align}
q(t)^2&=
\norm{a+t(b-a)}^2=
t^2\norm{b-a}^2+2ta'(b-a)+\norm a^2\\&=
t^2\norm{b-a}^2+2ta'(b-a)+\norm a^2
\end{align}
%
To see that this is increasing it will suffice to show that $a'(b-a)\geq 0$, that is
$$
\norm a^2\leq a'b=b'a.
$$
Now $\norm a^2 = t_*^2\norm g^2$ where $t_*\leq \norm g^2/(g'Hg)$ and so
$$
\norm a^2 \leq t_*\frac{\norm g^4}{g'Hg}
$$
Moreover
$$
b'a = t_*b'(-g)=t_*b'Hb.
$$
so that it will suiffice to show that
$$
\frac{\norm g^4}{g'Hg}\leq b'Hb.
$$
Recalling that $\norm g^2=g'g=-g'Hb$ we can rewrite this as
$$
(g'Hb)^2\leq (g'Hg)(b'Hb)
$$
which is the Cauchy-Schwartz inequality for the iner product $(g,b):=g'Hb$.
This shows (a). To see (b) compute the derivative $h'(t)$ as
%
\begin{align*}
h'(t) &:= 
\frac{d}{dt}f(a+t(b-a)\\&=
\frac{d}{dt}\big[g'(a+t(b-a)+\tfrac{1}{2}(a+t(b-a))'H(a+t(b-a))\big]\\&=
g'(b-a)+(a+t(b-a))'H(b-a)\\&=
g'(b-a)+a'H(b-a)+t(b-a)'H(b-a)
\end{align*}
%
We need to show that $\rho(t)\leq 0$, for $t\in[0,1]$. Since $(b-a)'H(b-a)\geq0$,
$h'(t)$ is increasing on this interval so this is equivalent with $h'(1)\leq 0$
which simplifies to
$$
g'(b-a)+b'H(b-a)\leq 0.
$$
Now for vectors $u$, $v$ the inner product $u'v$ is equal to $v'u$ and using this,
the symmetry of $H$ and $Hb=-g$ we have $b'H(b-a)=(b-a)'Hb=-g'(b-a)$ so that the above
sum is actually equal to zero.\qed

\medskip\noindent
We can now summarize the dog leg method as follows

\medskip\noindent
\textbf{Dog Leg Method.}
Find an approximate minmizer $p$ of $f(p)=g'p+\tfrac{1}{2}p'Hp$ subject to 
$\norm p\leq r$ as follows:
%
\begin{packed_itemize}
\item
Solve $Hb=-g$ (the global minimizer). If $\norm b\leq r$ set $p=b=p^H$.
\item
If $\norm b>r$ compute the Cauchy point $p^C_r$ as in (\ref{CauchyPoint_1}).
\item
If $\norm{p^C_r}=r$, equivalently if $t_*=r/\norm g$, set $p=p^C_r$.
\item
If $\norm{p^C_r}<r$, let $p$ be the dog leg point $p=p^D_r$ from (\ref{DoglegPoint}).
\end{packed_itemize}
%

\medskip\noindent
If the Cauch point satisfies $\norm{p^C_r}<r$ then the dog leg point $p^D_r$ will reduce
the value of the quadratic function $f$ more than the Cauchy point.

If however this quadratic function is the second order Taylor polynomial approximation of
a general nonlinear function $g$ in a neighborhood $B_r(x_k)$ of an iterate $x_k$, then 
the dog leg point $x_k+p^D_r$ is \textit{not guaranteed} to reduce the value of $g$ more  
than the Cauchy point $x_k+p^C_r$ and will do so only if the quadratic approximation $f$
of $g$ on $B_r(x_k)$ is accurate enough.

Generally this can be guaranteed only by making the radius $r$ small enough. Iterative methods
for minimizing a general nonlinear function $g$ then use heuristics to reduce or enlarge the 
\textit{trust radius} $r$ in each step according as the reduction in the value of $g$ in the 
last step was a satisfactory multiple of the reduction of the quadratic approximation $f$
of $g$.

If the global minimizer $b=-H^{-1}g$ is outside the ball $B_r(0)$ then the Cauchy- or dog leg 
point $p$ are guaranteed to satisfy $\norm p=r$. This is a consequence of the Lemma above.
This ensures that trust region iterative methods do not take steps that are too short and is  
a significant advantage over line search methods.





\medskip\noindent
\textbf{Regularization of the dog leg method.}
\label{sec:dog_leg_regularized}

Recall that the dog leg method is applied in the context of an iterative 
minmization of some $C^2$-function $f$. At the current iterate $x_k$ we 
use the second order Taylor polynomial $f$
$$
f(x_k+p)\simeq q(p)=f_k+g'p+p'Hp,
$$
where $f_k=f(x_k)$, $g=\nabla f(x_k)$ is the gradient and 
$H=D^2f(x_k)$ the Hessian of $f$ at the iterate $x_k$.

The radius $r$ defines the region $\norm p\leq r$ on which we believe that
the quadratic approximation $f(x_k+p)\simeq q(p)$ is sufficiently accurate
so that a step $p$ that decreases the value of $q$ will also lead to a 
sufficient decrease in the value of $f$ if we move from the iterate $x_k$ 
to the next iterate $x_{k+1}=x_k+p$. The ball $B_r(x_k)$ is the called the 
\textit{trust region} around $x_k$.

Thus we can compute the step $p$ by minmizing the quadratic function $q$ on 
the ball $\norm p\leq r$ and we are in the situation described above where the
dog leg method applies. 

Suppose now that we fix some $\l\geq 0$ and replace the matrix $H$ with 
$H+\l I$. This replaces the quadratic objective $q$ with the new function
%
\begin{equation}
\label{f_lambda}
q_\lambda(p)=g'p+p'(H+\l I)p = q(p)+\l\norm p^2.
\end{equation}  
%
and leaves the Cauchy point unchanged or moves it closer to the iterate $x_k$
along the line $x_k-tg$, see (\ref{CauchyPoint_1}), the value of $t_*$ may be 
decreased via the second term in the $\min$.

Instead of $p^H$ we now compute the point $p^H_\l:=p^{H+\l I}$ as the solution
of
%
\begin{equation}
\label{H_lambda}
(H+\l I)p=-g
\end{equation}  
%
and we know from proposition \ref{prop:f_eps_min} that this point is the absolute
minimizer of $q$ on the ball $\norm p\leq r(\l)$, where
$r(\l):=\norm{p^H_\l}$. From (\ref{norm_r}) we know that $r(\l)$ is a decreasing
function of $\l\geq 0$. 

To keep an iterative algorithm efficient we do not want to also solve the 
equation $Hp=-g$ for the point $p=p^H$. There are now several cases:

\medskip\noindent
\textbf{Case (I)} $r(\l)<<r$. This is the bad case as it would lead to a very 
small step $p$ and has to be avoided by decreasing $\l$. This case will however 
be unavoidable if the global minimum of $q$ is located close to the point $p=0$,
i.e. if the iterate $x_k$ above is already close to a local minimizer of $f$.

\medskip\noindent
\textbf{Case (II)} $r(\l)\leq r$ and $r(\l)\simeq r$. This is the ideal case
as the point $p^H_\l$ is nearly the global optimizer of $q$ on the ball
$\norm p\leq r$. 

\medskip\noindent
\textbf{Case (III)} $r(\l)>r$. This is the interesting case. We have two 
subcases.

\medskip\noindent
\textbf{Case (IIIa)} the Cauchy $p^C_r(\l)$ point is on the boundary $\norm p=r$.\\ 
In that case the algorithm takes the step $p=p^C_r(\l)$ and a step of sufficient size 
assured.  

\medskip\noindent
\textbf{Case (IIIb)} the Cauchy point $p^C_r(\l)$ satisfies $\norm{p^C_r(\l)}<r$.\\
Set $\delta:=r-\norm{p^C_r(\l)}$. We know from the proof of lemma \ref{lem:dogleg}
(applied to the matrix $H+\l I$ instead of $H$) that the function
$$
q_\l(p)=q(p)+\l\norm p^2
$$
decreases on the line segment $p=p(t)$ from the Cauchy point $p^C_r(\l)$ to the
dog leg point $p^D_r(\l)$ (computed with the matrix $H+\l I$). But since
$\norm p$ increases by the amount $\delta$ along the same line segment of the
dog leg path, the quadratic approxiation $q$ of $f$ must decrease at least by
the amount $\l\delta^2$ from the value at the Cauchy point along the same line
segment. 

\noindent
In other words we have
$$
q(p^D_r(\l))\leq q(p^C_r(\l))-\l\delta^2\leq q(0)-\l\delta^2.  
$$  
Now such a decrease must be small if the global minimizer of $q$ is 
close to zero since then the value of $q$ cannot be reduced significantly from
the value $q(0)$. 

This will occur if the iterate $x_k$ is close to a local minimizer 
of $f$ above and the radius $r$ is small enough so that the quadratic
approximation $f(x_k+p)\simeq q(p)$ is accurate.
Thus, as we approach a local minimizer of $f$ and shrink the trust radius $r$ 
below the distance to the minimizer the Cauchy point will approach the boundary 
$\norm p=r$. 

Replacing $H$ with $H+\l I$, where $\l>0$, has the advantage that now $H$ can
be singular and thus only needs to be positive semidefinite. The discussion 
above shows that that will not destroy the iterative algorithm as long as we 
ensure that $r(\l)$ is not too small and reduce $\l\downarrow 0$ as we approach
a local minimum, i.e. as the norm of the gradient $g$ approaches zero.

Now if $H=UDU'$ is the eigenvalue decomposition of $H$ with eigenvalues 
$0\leq \l_1\leq\l_2\leq\dots\leq\l_n$ and corresponding eigenvectors
$u_j=row_j(U)$ satisfying $Hu_j=\l_ju_j$ we have seen above that
%
\begin{equation}
\label{norm_r_lambda}
\norm{r(\l)}^2=\sum_j\frac{a_j^2}{(\l+\l_j)^2},
\end{equation}
%
where the $a_j=g'u_j$ are the components of the gradient $g$ in direction of
the eigenvectors $u_j$ of the Hessian $H$. In particular we have
$$
\sum_ja_j^2=\norm g^2.
$$ 
Unfortunately these components $a_j$ are unknown as it
is far too costly to compute the eigendecomposition of $H$. Recall that we
want to choose $\l$ such that
%
\begin{equation}
\label{lambda_eq}
\norm{r(\l)}=r
\end{equation}
% 
but we cannot precisely control the norm $r(\l)$ from (\ref{norm_r_lambda})
since the eigenvalues $\l_j$ and components $a_j$ are unknown. However we do 
know something about the eigenvalues of $H$: we have $\l_j\geq 0$ and the sum
$$
\sum_j\l_j = tr(H)
$$
is the trace of $H$. Recall that the trace satisfies $tr(AB)=tr(BA)$ and so 
$tr(H)=tr(UDU')=tr(DU'U)=tr(D)=\sum_j\l_j$). With this we can substitue the
average value 
$$
\mu:=\frac{1}{n}\sum_j\l_j=tr(H)/n
$$ 
for all the $\l_j$ into (\ref{norm_r_lambda}) to obtain the approximate
equation
%
\begin{equation}
\label{lambda_eq_1}
r=\norm{r(\l)}^2\simeq\sum_j\frac{a_j^2}{(\l+\mu)^2}=
\frac{1}{(\l+\mu)^2}\norm g^2.
\end{equation}
%
Solving for $\l$ yields 
%
\begin{equation}
\label{lambda_1}
\lambda=\norm g/\sqrt r-\mu
\end{equation}
%
which is useful only if $\l>0$ and gives a rough indication for a useful value
of $\l$. Note that this also indicates that $\l$ should shrink with the size of
the gradient $g$. Recall that we have already motivated above that we should move 
$\l\downarrow 0$ as $\norm g\downarrow 0$. This indicates the following heuristics 
for $\l$:
%
\begin{equation}
\label{lambda_2}
\lambda=\kappa\norm g/\sqrt r,
\end{equation}
%
where $\kappa\in(0,1)$ e.g. $\l=0.1$.

\medskip\noindent
\textbf{Remark.}
In any minimization algorithm the various points which we compute are all minimizers of 
the quadratic Taylor approximation $q$ to the objective function $f$ centered at the current 
iterate $x_k$. We have verified above that they will provide a satisfactory decrease in the 
value $q$ but not necessarily in the value of $f$.

A decrease in the value of $q$ implies a decrease in the value of $f$ only if $q$ approximates 
$f$ sufficiently closely on the ball $B_r(x_k)$. Clearly among the candidates for the next iterate
which we have computed:
%
\begin{packed_enum}
\item global minimizer $x_k+p^H$ of $q$,
\item Cauchy point $x_k+p^C_r$,
\item dog-leg point $x_k+p^D_r$ and
\item global minimizer $x_k+p^H(\l)$ of $q(x_k+h)+\l\norm h^2$, i.e. the solution of\\ $(H+\l I)p=-g$, 
\end{packed_enum}
%
We will choose as the next iterate the one 
that provides the most decrease in the value of $f$ and not in the value of $q$.

With this information we then also adjust the trust radius $r$. Suppose for example that the   
global minimizer $x_k+p^H$ of $q$ is inside the ball $B_r(x_k)$ but does not provide the best decrease   
in the value of $f$ (while it is guaranteed to provide the best decrease in the value of $q$).

Then the conclusion must be that $q$ does not approximate $f$ satisfactorily on all of $B_r(x_k)$  
and the trust radius $r$ needs to be decreased (note that the Cauchy point and dog-leg point both
depend on the radius $r$).

The dog-leg point is always on the boundary of the ball $B_r(x_k)$ and is not a desirable next iterate   
if $x_k$ is already very close to a local minimizer $x^*$ of $f$. In this case the gradient $g=\nabla f(x_k)$    
is close to zero hence the solutions $p^H$ of $Hp=-g$ and $p^H(\l)$ of $(H+\l I)p=-g$ will also be close to  
zero so that the points $x_k+p^H$ and $x_k+p^H(\l)$ are expected to be close to $x^*$. In fact if we keep
$\l>0$ bounded away from zero this is guaranteed to be true.

This suggests the following mixed strategy: in each step compute the Cauchy point $x_k+p^C_r$, the dog leg point
$x_k+p^D_r(\l)$ and the point $x_k+p^H(\l)$, where 
$$
\l = \max\{\ths\norm g/\sqrt r-\mu,\ths\kappa\ths\}
$$
for some small constant $\kappa>0$. The dog leg step $p^D_r(\l)$ is computed for the matrix $H+\l I$ i.e. we move 
in the direction $x_k+p^H(\l)$ from the Cauchy point $x_k+p^C_r$. With this strategy we need to solve only one
equation $(H+\l I)p=-g$ in each step entailing only one expensive matrix factorization.
This approach has the advantage that it can handle singular Hessians $H$ and has the further advantage that the 
regularized global minimizer $x_k+p^H(\l)$ is more likely to be inside the trust region $B_r(x_k)$ than the pure   
global minimizer $x_k+p^H$.


\subsection{Trust region versus backtracking line search}

To compute the step $p$ from the current iterate $x$ to the next iterate $x+p$ we minimize the quadratic Taylor 
approximation 
$$
\ol f(x+p)=f(x)+g\cdot p +\frac{1}{2}p'Hp
$$
as a function of the step $p$, where $g=\nabla f(x)$ is the gradient and $H=D^2f(x)$ is the Hessian of the objective function $f$ at $x$. This leads to the equation
%
\begin{equation}
\label{p_H}
Hp=-g,
\end{equation}
%
for the step $p$ and we called the solution the \textit{full Newton step}, denoted $p=p^H$.
However we do not expect this step to be optimal for a local minimization of the objective function $f$ itself,
since the quadratic approximation $\ol f$ of $f$ centered at the current iterate $x$ cannot be trusted to be  
a good approximation of $f$ globally.

The trust region method tries to identify a spherical region $B_r(x)$ around the current iterate $x$ on which this 
approximation can be trusted and computes the step by minimizing $\ol f(x+p)$ on the ball $B_r(x)$ instead. 
The problem is the identification of the proper trust radius $r$ and, given $r$, carrying out the minimization,
which is nontrivial.

In this regard we have found the following: if $\l>0$ and $p(\l)$ is the solution of the regularized Newton equation    
%
\begin{equation}
\label{p_H_l}
(H+\l I)p=-g,
\end{equation}
%
then $x+p(\l)$ minimizes the quadratic approximation $\ol f(x+p)$ of $f$ centered at $x$ on the ball $B_{r(\l)}(x)$,
where the radius $r(\l)$ is given by $r(\l)=\norm{p(\l)}$. See proposition \ref{prop:f_eps_min}.

\noindent
If we have a reasonable idea of a useful trust radius $r$ the problem now becomes that of solving the equation
%
\begin{equation}
\label{r_l}
r(\l)=r
\end{equation}
%
for the unknown $\l>0$. This is a nontrivial problem however. We have seen above that it can be solved using the 
eigenvalue decomposition of $H$ and expanding the gradient $g$ in terms of the eigenvectors of $H$. We have found 
that
$$
r(\l)^2 = \sum_j\frac{a_j^2}{(\l_j+\l)^2},
$$
where $Hu_j=\l_ju_j$ and $a_j=g\cdot u_j$ is the coefficient of $u_j$ in the expansion of $g$ in terms of the $u_j$.
With this we could in principle solve the equation $r(\l)^2=r^2$ but that approach is too expensive since the eigenvector   
decomposition of $H$ is expensive.

There is a cheaper approach. The problem is to find a way to cheaply solve the equation (\ref{p_H_l}) for many values   
of $\l$. The eigenvalue decomposition $H=UDU'$ with orthogonal $U$ and diagonal $D=diag(\l_j)$ allows us to do this   
since it implies the decomposition
$$
H+\l I=U(D+\l I)U'
$$
with the same orthogonal matrix $U$ and trivially modified diagonal matrix $D+\l I$. Basically this reduces the problem
to the solution of the trivial equation $(D+\l I)z=-U'g$ followed by a multiplication $p(\l)=Uz$. These operations are of 
order $O(n^2)$ while the eigenvalue decomposition is of order $O(n^3)$ with a large constant.

However we do not need to transform $H$ to a diagonal matrix with orthognal similarity, the same trick applies if we
transform $H$ to an upper triangular matrix, i.e we write $H=QRQ'$, where $R$ is upper triangular and $Q$ is orthogonal.
This is the so called \textit{Schur decomposition} of $H$, requires about $(8/3)n^3$ flops and is thus much cheaper
than the eigenvalue decomposition of $H$.

It reveals the eigenvalues of $H$ (as the diagonal elements of $R$) but we do not get the eigenvectors. With this we do
not have a simple formula for $r(\l)$ as above but we still have the relation
$$
H+\l I = Q(R+\l I)Q'
$$
for all $\l$ and so we can reduce the solution of the equation $(H+\l I)p=-g$ to the solution of $(R+\l I)z=-Q'g$
followed by $p(\l)=Qz$ which impies that $\norm{p(\l)}=\norm z$. The matrix $R+\l I$ is still upper triangular, 
so the equation $(R+\l I)z=-Q'g$ can be solved cheaply ($n(n+1)/2$ flops, fewer than the matrix-vector product 
$Q'g$) so that an iterative approach to the solution of the equation $\norm{z(\l)}=r$ becomes practical.

Let us note in particular that one single Schur decomposition is enough to handle all values of $\l$. But the Schur 
decomposition is still expensive compared to the Cholesky decomposition which only takes $n^3/3$ flops. Thus an 
iterative algorithm has been devised which is based on the Cholesky decomposition. However in this algorithm we have
to carry out a new Cholesky factorization of $H+\l I$ for each new value of $\l$. Still, since in general fewer than
8 iterations are needed, this is the most economical approach, see the document \cite{NG}.

Let us note that all these approaches are incompatible with a Ruiz-rebalancing of the matrix $H$
since this will imply a final multiplication of the solution with the diagonal balancing matrix
$D$ which alters the norm in uncontrollable ways. We do not need this rebalancing of $H$ however if
we know the eigenvalues of $H$, since we then know the effect of $\l$ on the condition number of
$H+\l I$.

All this is complicated and still does not resolve the question of determining a useful trust radius $r$. In practice
this is handed by starting with a reasonbable guess and adjusting $r$ in each step based on the performance in the last step.
This is imperfect and in particular leaves the following question: 

\textit{why would a spherical region $B_r(x)$ be preferred as
a trust region for the quadratic approximation $\ol f$ of $f$ centered at $x$?}

In fact a reasonable trust region has the form $|f(x+p)-\ol f(x+p)|<\epsilon$ and it can have any shape whatsoever,
certainly it does not need to be spherical. Given the difficulties with spherical trust regions outlined above this is
certainly a reasonable question to ask.

Now let us recall the backtracking line search. In this approach we simply minimize the objective
function $f$ on the line segment
$$
[x,x+p^H]=\{\ths x+t p^H\ths:\ths t\in[0,1]\ths\}.
$$
I.e. we make the full Newton step $p^H$ and backtrack from there in a straight line. This is certainly a much
simpler approach. In this regard it is interesting to note that backtracking line search can also be motivated 
via trust regions of a different shape, namely of the form
$$
D_r(x) = \{\ths x+p\ths:\ths p'Hp\leq r\ths\}.
$$
This is an elliptical region centered at $x$ defined by the Hessian $H$ of $f$ at $x$. Suppose now we try to minimize
the quadratic approximation $\ol f$ on $D_r(x)$ by adding a penalty term $\frac{\l}{2}p'Hp$ to $\ol f(x+p)$ (for some 
fixed value $\l>0$) and minimizing $\ol f$ plus the penalty term globally, i.e. minimize the function
$$
\ol f_s(x+p) = \ol f(x+p)+\frac{\l}{2}p'Hp = f(x)+g'p+\frac{1+\l}{2}p'Hp
$$
as a function of the step $p$ leading to the equation
$$
(1+\l)Hp=-g.
$$
Clearly the solution $p(\l)$ is simply given by
$$
p(s)=\frac{1}{1+\l}p^H
$$
i.e. corresponds to a backtracking operation on the line segment $[x,x+p^H]$. Now exactly the same argument as
in the proof of proposition \ref{prop:f_eps_min} shows that
%
\begin{prop}
\label{prop:back_track}
The step $p(\l)=(1+\l)^{-1}p^H$ minimizes the quadratic approximation $\ol f(x+p)$ on the region
$p'Hp\leq r(\l)$ where $r(\l)=p(\l)'Hp(\l)$.
\end{prop}
%
\textbf{Proof.} $p(\l)$ is the global minimizer of $\ol f(x+p)+(\l/2)p'Hp$. Now assume that
$p'Hp\leq p(\l)'Hp(\l)$. Then
$$
\ol f(p(\l))+\frac{\l}{2}p'Hp \leq 
\ol f(p(\l))+\frac{\l}{2}p(\l)'Hp(\l)\leq
\ol f(p)+\frac{\l}{2}p'Hp
$$
which implies that $\ol f(p(\l))\leq\ol f(p)$.\qed

As in the case of spherical trust regions this concerns only the minimum of the quadratic approximation
$\ol f$ of $f$ not the objective function $f$ itself and leaves open the question of where this approximation
can be trusted. But there is no reason to prefer spherical trust regions to elliptical ones.

In this case however this entire question becomes irrelevant since we can simply minimize the objective 
function $f$ itself on the line segment $[x,x+p^H]$ which can be done by a line search algorithm.

One idea is to mix the two approaches (backtracking line search and spherical trust region): first
compute $p(\l)$ by backtracking line search (exact optimization), then define the trust radius $r$
as $r=\norm{p(\l)}$ and minimize on the spherical trust region $B_r(x)$ with one of the approaches 
above.

\begin{example}
\label{example:rosenbrook}
As an example let us consider the notorious Rosenbrook function
%
\begin{equation}
\label{rosenbrook_fn}
f(x,y)=(x-a)^2+b(y-x^2)^2
\end{equation}
%
The global minimum is at $(x,y)=(a,a^2)$ and the function is nasty if $b$ is big, e.g. $a=1$, $b=100$.
We have
%
\begin{align*}
f_x &= 2(x-a)-4b(y-x^2)x,\q f_y=2b(y-x^2),\q\text{and thus}\\
f_{xx} &= 2-4b(y-3x^2),\q f_{xy}=f_{yx}=-4bx,\text{ and }f_{yy}=2b.
\end{align*}
%
The function itself is a polynomial and the quadratic Taylor approximation $\ol f(h,k)$ of $f(x+h,y+k)$ contains
all the second order terms in $h$, $k$ while the residual $f(x+h,y+k)-\ol f(h,k)$ contains the rest and is easily 
seen to be the quantity
$$
f(x+h,y+k)-\ol f(h,k) = h^2(bxh+h^2-2bk)
$$
Thus reasonable trust regions for this function have the form
$$
|h^2(bxh+h^2-2bk)|\leq\epsilon
$$
and are neither spherical nor elliptical.\qed
\end{example}
%




